\documentclass{report}

\usepackage{graphicx} % Add pictures to your document
\usepackage{float}

\begin{document}

\title{Formulário de CGRA}
\maketitle{}

\chapter{Modelos de iluminação local}
\textbf{Reflexão difusa} - a luz reflete em todas as direções com intensidade definida pela rugosidade da superfície.\\
\textbf{Reflexão especular} - depende da direção de observação.\\
\textbf{Lei de Lambert} - a intensidade da luz refletida depende do ângulo de iluminação.\\
\textbf{Atenuação atmosférica} - dada pela distância do observador ao objeto.\\

\begin{equation}
    I = I_a \times K_a + I_p * \left(\frac{K_d}{d + d_0} \times N.L + \frac{K_s}{d + d_0} \times (R.V)^n\right)
\end{equation}

\begin{itemize}
    \itemsep0em 
    \item $K$'s, variam de 0 a 1 e dependem do material;
    \item $N$, normal à superfície;
    \item $L$, direção da fonte de luz;
    \item $R$, direção da reflexão perfeita;
    \item $V$, direção do observador;
    \item $V$, direção do observador;
    \item $n$, depende da superfície. 1 para faces não polidas, 200 para faces perfeitamente polidas.
    \item $d + d_0$, fator de atenuação. Se não for especificado, é 1. O real seria $d^2$ mas produz maus
    resultados com a aproximação da iluminação ambiente.
\end{itemize}

\chapter{Shading and smooth shading}
\section{Sombreamento constante}
Calcula-se a iluminação local num ponto que der jeito (e.g.: vértice conhecido) e dizemos que
os restantes pontos dessa face tem a mesma iluminação.\\
\textbf{Efeito de Mach Band} - variação brusca de iluminação faz as zonas claras parecerem
mais claras (e vice-versa).

\section{Método de Gouraud}
\begin{enumerate}
\item Calcular a cor de cada vértice através do modelo de iluminação pretendido.
\item Calcular a cor dos restantes pontos do polígono por interpolação bi-linear.
\end{enumerate}
\textbf{Interpolação bi-linear} - primeiro vertical depois horizontal.\\

\begin{equation}
    I_p = I_b - (I_b - I_a) \times \frac{y_b - y_p}{y_b - y_a}
\end{equation}

\textbf{Efeito de Mach Band} - existe nas descontinuidades da derivada. Convexidades sãoexageradas.\\
\textbf{Nota} - não consegue ir buscar valores máximos de luz se tiverem no centro dos polígonos.

\section{Método de Phong}
Fazemos a interpolação bi-linear das normais na face (ficam aproximadamente iguais às reais).
Consegue ir buscar valores máximos de luz que se encontram no centro de faces.\\
\textbf{Efeito de Mach Band} - praticamente inexistente com este método.

\chapter{Textures}
\section{Mapeamento de texturas 2D}
\begin{itemize}
\item Um pixel numa textura é denominado por \textbf{texel}.
\end{itemize}

\chapter{Modelação de Sólidos}

\begin{equation}
    V – E + F – H = 2 (C – G)
\end{equation}

\begin{itemize}
    \item $V$, vértices
    \item $E$, arestas
    \item $F$, faces
    \item $H$, número de buracos nas faces
    \item $G$, número de buracos que atravessam o objeto
    \item $C$, número de partes do objeto
\end{itemize}

\end{document}